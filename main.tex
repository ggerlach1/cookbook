\documentclass{article}
\usepackage[utf8]{inputenc}
\usepackage{geometry}
\usepackage{nicefrac}
\usepackage{gensymb}
\usepackage[nowarnings]{xcookybooky}
\usepackage{imakeidx}
\usepackage{float}


\usepackage{hyperref} % this must be the last package that is imported!

\newcommand{\faren}{\degree F }
\renewcommand{\step} % fixing the error with xcookybooky somehow https://tex.stackexchange.com/questions/481698/missing-endcsname-in-package-xcookybooky-texlive
{%
 \stepcounter{step}%shouldn't be in the argument of lettrine
    \lettrine
    [%
        lines=2,
        lhang=0,          % space into margin, value between 0 and 1
        loversize=0.15,   % enlarges the height of the capital
        slope=0em,
        findent=1em,      % gap between capital and intended text
        nindent=0em       % shifts all intended lines, begining with the second line
    ]{\thestep}{}%
}
\geometry{
 letterpaper,
 left=1in,
 top=1in,
 }

\title{Recipes Not easy to print}
\author{Gaby Gerlach}
\date{\today}
\makeindex[intoc]

\begin{document}
\maketitle
\tableofcontents

\newpage
\section{Desert}
\begin{recipe}[source=Claire Saffitz]{Pecan Brittle Oatmeal Cookies}

\ingredients[15]{%
& \textbf{Pecan Brittle}\\
\unit[142]{g} & Coarsely chopped pecans \\
\unit[150]{g} & granulated sugar\\
\unit[57]{g} & unsalted butter \\
\unit[1/2]{tsp} & baking soda\\
\unit[1/2]{tsp} & kosher salt\\
& \textbf{Cookies}\\
\unit[2]{sticks} & unsalted butter, cut into tbsp\\
\unit[173]{g} & all-purpose flour\\
\unit[2]{tsp} & kosher salt\\
\unit[1]{tsp} & baking soda \\
\unit[200]{g} & old-fashioned rolled oats\\
\unit[150]{g} & dark brown sugar\\
\unit[100]{g} & granulated sugar\\
\unit[2]{large} & eggs (cold from the fridge)\\
\unit[1]{tbsp} & vanilla extract\\
}

\preparation{% 
\\
\textit{Make the Pecan Brittle}
\step Preheat oven to 350 \faren. Coarsely chop the pecans and bake until fragrant 8-10 minutes. Do not burn nuts!
\step Combine salt and baking soda in a small container set near stove. Place silicon baking mat on baking sheet, set aside.
\step Combine sugar and butter in sauce pan over medium heat with ~2 tbsp of water. Stir until melted than switch to swirling the pan (this prevents crystal formation). 
\step Boil until the sugar/caramel reaches a deep golden. Turn off the heat and working quickly stir in the pecans. Finally add the  baking soda/salt mixture and stir until combined.
\step Pour the mixture onto the baking mat and flatten to a single layer while warm. Set aside too cool.
\step Once cool and solid break it into large chunks using your hands and then cut so the largest pieces are about the size of a pea (it doesn't have to be neat). Set aside.
\\
\textit{Make the batter}
\step Make brown butter from ONE of the sticks of butter (you are only browning half of the butter)
\step Pour the browned butter into the bowl of your stand mixer with the other stick of butter. This now needs to cool to room temperature.
\step Put the flour, baking soda, half of the oats, and half of the pecan brittle into the food processor and pulse until you have formed a flour. 
\step Add the brown and white sugar to the mixer and combine. Add the cold eggs one at a time and combine. Add the flour mixture and carefully combine.
\step Scoop 2 once (1/4 cup) blobs onto a baking sheet and refrigerate for 24 hours.
\step Preheat oven to 350 \faren. Place 6 dough balls on a lined baking sheet (they spread) and bake for 16-20 minutes rotating once after 12 minutes. 
}

\hint{
You can freeze the dough balls after they have been in the fridge for a delicious single cookie whenever you feel like turning on the oven!
}
\end{recipe}

\begin{recipe}[source = Laura Howerton]{Ginger Cookies}
\ingredients[15]{
\unit[4]{cups} & All-purpose flour \\
\unit[2]{sticks} & unsalted butter \\
\unit[2]{tsp} & baking soda \\
\unit[4]{tbsp} & ground ginger \\
\unit[1]{tbsp} & ground clove \\
\unit[1]{tbsp} & nutmeg \\
\unit[2]{tbsp} & cinnamon \\
2 & eggs\\
\unit[1/2]{cup} & molasses \\
\unit[1]{cup} & granulated sugar \\
\unit[1]{cup} & brown sugar\\
}

\preparation{
\newline
\step Pre-heat oven to 350 \faren.
\step In a medium bowl stir together flour, ginger, baking soda, cinnamon, cloves, and salt; set aside.
\step In a large mixing bowl beat butter for 30 seconds, add the brown and white sugar, beat until combined. Beat in eggs and molasses until combined.
\step Add the dry ingredients to the wet ingredients.
\step Shape dough into 1-inch balls, place on cookie sheet and bake for 8 to 9 minutes until just done with the tops puffed. Cool on a wire rack.
}
\end{recipe}
\newpage
\section{Yeast-y baking}
\begin{recipe}[source=Bob's Red Mill Bakery]{Rye Bread}

\ingredients[15]{
\unit[2 1/4]{tsp} & active dry yeast \\
\unit[1 1/4]{cups} & warm water \\
\unit[1 1/2]{tsp} & molasses \\
\unit[1]{tbsp} & Oil \\
\unit[1 3/4]{cups} & bread flour \\
\unit[1]{cup} & dark rye flour \\
\unit[2]{tbsp} & vital wheat gluten \\
\unit[1]{tbsp} & caraway seeds \\
\unit[1 1/2]{tsp} & salt \\
}

\preparation{
\newline
\step Sprinkle yeast over water and molasses in a large mixing bowl and let sit for 5 minutes. Add remaining ingredients and mix until dough pulls away from the sides of the bowl. Turn dough onto a lightly counter and knead for about 10 minutes, or until you can stretch a small portion of the dough into a thin membrane. I was not able to knead enough with an electric mixer.
\step place dough in a clean oiled bowl. Cover and allow to rise until doubled. Punch down dough, cover and let rise another 15 minutes. Preheat oven to 350 \faren and lightly oil a loaf pan.
\step place dough on a lightly floured counter, shape into a loaf and place in prepared pan. cover and let rise for about 1 hour or until the dough crowns above the pan and gives with a gentle press of the fingers, leaving a faint indentation.
\step Bake for 30 minutes or until golden-brown and hollow sounding when tapped. Cool on a wire rack.
}
\end{recipe}
\newpage
\section{Holiday favorites}
\begin{recipe}[%
preparationtime={\unit[1.5]{hours}},
portion=\portion{8-10},
source=Food\&Wine Magazine
]
{Sweet Potato Biscuits}

\ingredients[15]{
1 & medium sized sweet potato\\
\unit[1 3/4]{cups} & all-purpose flour\\
\unit[1]{tbsp} & brown sugar \\
\unit[2 1/2]{tsp} & baking powder \\
\unit[1/2]{tsp} & baking soda \\
\unit[1]{tsp} & salt \\
\unit[7]{tbsp} & cold unsalted diced butter \\
\unit[1/3]{cup} & butter milk \\
}

\preparation{
\newline
\step Preheat the oven to 375 \faren. Poke the sweet potato all over with a fork and bake for about 45 minutes, until tender. Peel and mash the potato. Set aside 3/4 cup of the mashed potato and let cool completely; reserve the rest for another use. Raise the oven temperature to 425 \faren.
\step In a food processor, pulse the flour with the brown sugar, baking powder, baking soda, and alt. Add the butter and pulse until the mixture resembles coarse meal. Add the buttermilk and the 3/4 cup mashed potato and pulse until the dough comes together.
\step Turn the dough out onto a heavily floured work surface and knead 2 or 3 times, until smooth; the dough will be soft. Roll out the dough 1/4 inch thick and cut out 8 4-inch rounds. A cup or mason jar works well here. Arrange the biscuits on a baking sheet. Bake for about 15 minutes, until golden brown. 
}
\hint{The sweet potato mashed can be made a couple days ahead and stored in the fridge. I almost always double this recipe because I have enough sweet potato}
\end{recipe}

\newpage
\begin{recipe}[
preparationtime= 1 hour 5 min,
source = Its a keeper blog]
{Thanksgiving Stuffing}
\ingredients[15]{
\unit[2]{loaves} & sandwich bread\\
\unit[12]{tbsp} & unsalted butter \\
4 & onions \\
1 & head of celery \\
\unit[2]{tbsp} & dried sage \\
\unit[1]{tbsp} & salt \\
\unit[1]{tbsp} & black pepper \\
\unit[1]{tbsp} & garlic powder \\
\unit[1]{tbsp} & onion powder\\
\unit[1/2]{cup} & vegetable broth \\
}

\preparation{
\newline
\step Cut the bread into \~ 1-inch cubes and sit out to dry over night.
\step Pre-heat oven to 350\faren. Dice the onions and celery. Melt butter in a large skillet
\step Add the onions, celery, and spices to the pan. Saute until vegetables are translucent and tender.
\step Grease a 13 x 9 baking dish. Place the bread in the dish and cover with the onion mixture and broth. Stir to combine add salt and pepper to taste. 
\step bake for about 30 minutes
}
\hint{I like to use a mixture of breads so one loaf of sourdough and one of a more fluffy bread. 2 loaves is a recommendation it depends heavily on the size of your loafs. Feel free to add more spices to taste, Rosemary and Italian seasoning are both a good addition as is some minced garlic.}
\end{recipe}


\end{document}


