
\section{Holiday favorites}
\begin{recipe}[%
preparationtime={\unit[1.5]{hours}},
portion=\portion{8-10},
source=Food\&Wine Magazine
]
{Sweet Potato Biscuits}

\ingredients[15]{
1 & medium sized sweet potato\\
\unit[1 3/4]{cups} & all-purpose flour\\
\unit[1]{tbsp} & brown sugar \\
\unit[2 1/2]{tsp} & baking powder \\
\unit[1/2]{tsp} & baking soda \\
\unit[1]{tsp} & salt \\
\unit[7]{tbsp} & cold unsalted diced butter \\
\unit[1/3]{cup} & butter milk \\
}

\preparation{
\newline
\step Preheat the oven to 375 \faren. Poke the sweet potato all over with a fork and bake for about 45 minutes, until tender. Peel and mash the potato. Set aside 3/4 cup of the mashed potato and let cool completely; reserve the rest for another use. Raise the oven temperature to 425 \faren.
\step In a food processor, pulse the flour with the brown sugar, baking powder, baking soda, and alt. Add the butter and pulse until the mixture resembles coarse meal. Add the buttermilk and the 3/4 cup mashed potato and pulse until the dough comes together.
\step Turn the dough out onto a heavily floured work surface and knead 2 or 3 times, until smooth; the dough will be soft. Roll out the dough 1/4 inch thick and cut out 8 4-inch rounds. A cup or mason jar works well here. Arrange the biscuits on a baking sheet. Bake for about 15 minutes, until golden brown. 
}
\hint{The sweet potato mashed can be made a couple days ahead and stored in the fridge. I almost always double this recipe because I have enough sweet potato}
\end{recipe}

\newpage
\begin{recipe}[
preparationtime= 1 hour 5 min,
source = Its a keeper blog]
{Thanksgiving Stuffing}
\ingredients[15]{
\unit[2]{loaves} & sandwich bread\\
\unit[12]{tbsp} & unsalted butter \\
4 & onions \\
1 & head of celery \\
\unit[2]{tbsp} & dried sage \\
\unit[1]{tbsp} & salt \\
\unit[1]{tbsp} & black pepper \\
\unit[1]{tbsp} & garlic powder \\
\unit[1]{tbsp} & onion powder\\
\unit[1/2]{cup} & vegetable broth \\
}

\preparation{
\newline
\step Cut the bread into \~ 1-inch cubes and sit out to dry over night.
\step Pre-heat oven to 350\faren. Dice the onions and celery. Melt butter in a large skillet
\step Add the onions, celery, and spices to the pan. Saute until vegetables are translucent and tender.
\step Grease a 13 x 9 baking dish. Place the bread in the dish and cover with the onion mixture and broth. Stir to combine add salt and pepper to taste. 
\step bake for about 30 minutes
}
\hint{I like to use a mixture of breads so one loaf of sourdough and one of a more fluffy bread. 2 loaves is a recommendation it depends heavily on the size of your loafs. Feel free to add more spices to taste, Rosemary and Italian seasoning are both a good addition as is some minced garlic.}
\end{recipe}
\newpage
\begin{recipe}[source = Allrecipes, preparationtime=15min]
		{Mulled Wine}
		\ingredients[7]{
				1 & bottle red wine \\
				\unit[2]{cups} & apple juice \\
				\unit[3]{tbsp} & honey \\
				3 & cinnamon sticks \\
				10 & cardamom pods \\
				8 & whole cloves \\
				\unit[1/2]{tsp} & anise seed \\
		}

		\preparation{
		\step Stir all ingredients together in a saucepan over medium heat.
		\step Cover. Simmer for at least 10 minutes but longer will just lead to more flavor.
		}
\end{recipe}
