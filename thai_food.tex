\section{Thai Food}
\begin{recipe}[source = tastes better from scratch]{Pad Thai Sauce}

		\ingredients[6]{
		\unit[3]{tbsp} & fish sauce \\
		\unit[1]{tbsp} & low-sodium soy sauce \\
		\unit[2]{tbsp} & brown sugar \\
		\unit[2]{tbsp} & rice vinegar \\
		\unit[1]{tbsp} & Sriracha hot sauce \\
		\unit[2]{tbsp} & peanut butter \\
		\unit[1]{tsp}  & hot pepper flakes \\
		}

		\preparation{
		\step Combine all ingredients, set aside or store in fridge until ready for use.}

\end{recipe}

\begin{recipe}[source=tastes better from scratch]{Massaman Curry}
		\ingredients[14]{
		\unit[2]{tbsp}  & olive oil \\
		1/2             & onion - chopped \\
		1               & package tofu \\
		2               & medium gold potatoes \\
		2               & carrots \\
		\unit[2]{tsp}   & freshly grated ginger \\
		2               & cloves garlic \\
		1               & 4 oz can massaman curry paste \\
		2               & 13.5 oz cans coconut milk \\
		\unit[1]{tbsp}  & peanut butter \\
		\unit[2]{tbsp}  & brown sugar \\
					    & juice from 1 lime \\
		\unit[2]{tsp}   & fish sauce \\
		\unit[2-5]{tsp} & chilly powder \\
		\unit[1/2]{cup} & chopped roasted peanuts \\
		}

		\preparation{
		\step Heat the oil in a large pot over medium low heat. Add onion and saute for 1 minute until softened. Add the chopped carrots and potatos and cook for a couple minutes. Add the tofu, ginger, garlic, and curry paste and saute for about 3 minutes.
		\step Add the coconut milk. Bring to a boil. Reduce hear and simmer for 10-15 minutes or until the tofu and potatoes are cooked through. Stir in fish sauce, brown sugar, peanut butter, chilly powder, and lime juice. Simmer for 5 more minutes.
		\step Serve over rice while hot!
		}
		\hint{This also freezes well}
\end{recipe}
