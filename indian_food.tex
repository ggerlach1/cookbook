\section{Indian Food}
\begin{recipe}[source = Abhi's Mom]{Peanut Chutney and Dosas}
		\ingredients[12]{
				& \textit{Dosas} \\
				\unit[1]{cup} & black gram dal \\ 
				\unit[3]{cups} & basmati rice  \\
				& \textit{Peanut Chutney} \\
				\unit[1]{cup} & peanuts \\
				3 & cloves garlic \\
				3-5 & dried chillies \\
				\unit[1]{tbsp} & cumin seeds\\
				\unit[1]{tbsp} & coriander seeds\\
				 & half pod of tamarin \\
							  & salt \\ 
				& ghee\\
		}
		\preparation{%
				\\
				\textit{Dosas}
				\step Combine the gram dal and rice rinse several times and cover with water enough that there are several inches of water above the rice. Allow to soak for 8 hours.
				\step Using a high powered blender, grind the rice and gram dal using a sparing amount of water. The goal it to have it be somewhere between crepe and pancake batter consistency. Cover and leave over night to ferment.
				\step In the morning if it looks bubbly place in the fridge until you are ready to use.
				\step Using a pan that is able to get very hot, place a small amount of ghee on the pan and lay out about 1/3-1/2 cup of batter. Use the back of the measuring cup to further spread the batter. Flip when it looks like the top side has mostly dried out. Cook for 2-3 minutes until the second side is crispy. Repeat. These are much better hot so try to time it with the chutney cooking.
				\\
				\textit{Peanut Chutney}
				\step Toast the peanuts in a pan until they begin to brown. Add the garlic cloves along with the cumin and coriander seeds. Cook for a minute or two until fragrant.
				\step Put the peanut spice mixture in the blender with the chillies and tamarin (you only want the meat part of the pod, so remove the seeds). Add enough water so you can blend it (start with about a cup) and salt. Blend until smooth the consistency should be that of hummus. Add a little (1-2 tbsp) of ghee and adjust spices to you liking.
				\step serve warm and eat with your hands!
		}

		\hint{The dosa batter will stay good in the fridge for about a week as will the cooked chutney. I like it with avocado.}
\end{recipe}

\begin{recipe}[source=Abhi's mom]{Paneer Masala}
		\ingredients[5]{
		5 & medium tomatoes \\
		2 & yellow onions-diced \\
		1 & package paneer \\
		\unit[4]{tbsp} & butter \\
		\unit[4-5]{tbsp} & gram masala \\
		\unit[2-3]{tsp} & chilly powder \\
						& salt \\}

		\preparation{
		\step Put diced onions in a large sauce pan with a little neutral oil and saute for a couple minutes.
		\step Add the diced tomatoes, cook for some time. The goal is to remove a lot of the water from both the onions and tomatoes, so how long depends on how wet all of your ingredients where.
		\step Use an emersion blender to blend the onions and tomatoes into a thick soup consistency. 
		\step Slice the paneer into cubes and cook it you can: boil it in water, fry it in oil, or air fry it for about 5 minutes on 350 \faren.
		\step add the spices and stir. Adjust spices to taste and add butter. 
		\step once butter is melted add the paneer. 
		\step serve hot over rice.
		}

		\hint{This stuff freezes really well!}

\end{recipe}
